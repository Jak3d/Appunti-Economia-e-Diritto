\documentclass[twocolumn]{article}
\setlength{\columnsep}{20pt}
\usepackage[italian]{babel}
\usepackage[letterpaper,top=2cm,bottom=2cm,left=3cm,right=3cm,marginparwidth=1.75cm]{geometry}

% Useful packages
\usepackage{amsmath}

\usepackage[skins]{tcolorbox}
\tcbset{commonstyle/.style={boxrule=0pt,sharp corners,enhanced jigsaw,boxsep=0pt,left=\fboxsep,right=\fboxsep}}
\newtcolorbox{mycolorbox}[1][]{commonstyle,#1}
\definecolor{colorone}{RGB}{220,220,220}
\newcommand{\definition}[1]{\begin{mycolorbox}[colback=colorone]
\fontfamily{qcr}\selectfont #1 \fontfamily{cmr}\selectfont
\end{mycolorbox}}

\usepackage[colorlinks=true, allcolors=black]{hyperref}
\usepackage[T1]{fontenc}
\usepackage[
    type={CC},
    modifier={by-nc-sa},
    version={3.0},
]{doclicense}
\usepackage{graphicx}
\usepackage{fancyhdr}
\usepackage{imakeidx}
\makeindex[columns=3, title=Alphabetical Index, intoc]
\hypersetup{ 
     colorlinks=true, 
     linkcolor=black, 
     filecolor=black, 
     citecolor = black,       
     urlcolor=blue, 
     } 
\title{Economia e Diritto}
\author{Leonardo Marro}
\date{March 2023}

\begin{document}



\twocolumn[{\centering{\Huge Economia e Diritto \par}\vspace{3ex}
        {\large Leonardo Marro \par}\vspace{3ex}
	\today\par\vspace{4ex}{\small \fbox{\doclicenseThis} \par}\vspace{2ex}}]
 %pagina separata? o insieme al titolo?
 \tableofcontents
\clearpage
\part{Economia}


\section{Introduzione}

\section{Customer Segments}

The customer segment building block defines the different groups of people or organizations an enterprise aims to reach and serve. 
\subsection{Monopòli}
Come per quelli dello stato sono certi prodotti o servizi che sono ottenibili da un solo ente, privato o pubblico.
\subsection{Distribution Services}
Quando un consumatore acquista da certe aziende, non si domanda \textbf{da chi viene prodotto}, osserva solo il distributore di tale prodotto (vedi Enel con elettricità).
\subsection{Requisiti delle compagnie}
Una compagnia potrebbe aver bisogno di distinguere i consumatori in diversi gruppi: \begin{itemize}
    \item Necessità comuni
    \item Comportamenti comuni
    \item Altri attributi
\end{itemize}
Un modello di business può creare diversi segmenti di consumatori di dimensioni diverse, è importante saper decidere \textbf{chi servire e chi ignorare}.

Quando queste decisioni vengono attuate, si può procedere a creare un modello di business incentrato alla comprensione di \textbf{necessità specifiche dei consumatori}.

Esempi di necessità:\begin{itemize}
    \item Pausa Pranzo: serve un posto dove prendere il cibo, un posto dove mangiare, qualcuno che pulisca il posto per mangiare.
\end{itemize}

Per capire quali sono le necessità dei consumatori serve un metodo di ottenimento dei dati in modo che sia efficiente ed affidabile. Un metodo utile è l'ottenimento degli utenti che accedono ad una certa struttura attraverso il \textbf{GPS dei cellulari}, con questi dati si può ottenere l'affluenza media, gli orari ed i giorni etc...

I gruppi di consumatori si separano se:\begin{itemize}
    \item I loro bisogni \textbf{richiedono o giustificano un'offerta distinta};
    \item Arrivano da \textbf{diversi canali di distribuzione};
    \item Richiedono diversi \textbf{tipi di relazioni};
    \item Hanno diversa \textbf{profittabilità};
    \item Sono disposti a pagare per diversi \textbf{aspetti dell'offerta}.
\end{itemize}
\subsubsection{Domanda Chiave}
\textbf{Per chi stiamo creando valore? \newline Chi sono i consumatori più importanti?}
\subsection{Tipi di segmenti}
\subsubsection{Mercato di Massa}
Si concentra su grandi gruppi che hanno gli stessi problemi e le  stesse necessità.
\subsubsection{Mercato di Nicchia}
Mirata a consumatori con specifiche necessità.
\subsubsection{Segmentato}
Diversi problemi che si differenziano di poco (tipo banca con operazioni legali etc.)
\subsubsection{Diversificato}
Gestisce consumatori con necessità e problemi notevolmente diversi.
\subsubsection{Piattaforme Multi-Sided}
[Non l'ho capito]
\section{Proposizione di Valore}
Qual è la ragione per cui sto servendo il consumatore? Risolve un \textbf{Problema}(In questo caso posso incrementare i guadagni, diversa propensione al prezzo/qualità) o soddisfa una \textbf{Necessità}?.
\subsection{Le 5 domande}
\begingroup
\begin{tabular}{X|c|c}
& \definition{Qual è lo scopo del prodotto?} &  \\
& \definition{A chi è mirato il prodotto?} & \\
& \definition{A quali necessità riferisce il prodotto?} & \\
& \definition{Come viene consegnato il prodotto?} & \\
& \definition{Qual è la posizione di prezzo del prodotto?} & \\
\end{tabular}
\endgroup
Queste sono domande da applicare a qualsiasi prodotto per poter studiare a quale segmento riferirlo.
\begingroup
\definition{\textbf{Cross-Selling}: Se vendo un certo oggetto nel negozio, posso vendere prodotti completamente diversi, ma che possono essere abbinabili al primo.}
\endgroup
Hai una proposizione di valore solo se:
\[C < P < V\]
\begingroup
\definition{C = Costo totale del prodotto} Esistono casi in cui le aziende vendono prodotti a prezzo inferiore del costo di produzione, questo perché così posso \textbf{soffocare la nuova competizione}; spesso questo è un comportamento delle grandi aziende. In certi stati questo è \textbf{illegale} (Anti-Trust). 
\definition{P = Customer Net Price} Valore per il cliente, a che punto un cliente è disposto a comprare il nostro prodotto? (Nota: Se non sei italiano non paghi le imposte!)
\definition{V = Customer Perceived Price}
\endgroup
\textbf{Un Modello di Business descrive il procedimento razionale di come un'organizzazione crea, consegna e cattura i valori}. Come creo, distribuisco o catturo(prendo pezzi da parti diverse e le rendo mie) il valore?
\subsection{Value Proposition}
\textbf{Il blocco di proposizione di valore descrive il bundle di prodotti e servizi che creano un valore per uno specifico segmento di valore }.
Una value proposition può essere di due tipi: \begin{itemize}
    \item Quantitativa (Prezzo più basso, velocità del servizio);
    \item Qualitativa (Design, qualità più alta).
\end{itemize}
\section{Descrizione degli Elementi}
\begin{itemize}
    \item Newness(novità):
    \item Performance:
    \item Customization(Personalizzazione):
    \item Completing the job:
    \item Design:
    \item Brand/Status:
    \item Price:
    \item Accessibility:
    
\end{itemize}


\clearpage\newpage
\part{Diritto}
\setcounter{section}{0}
\textbf{Prima di iniziare}: in questa materia la terminologia è estremamente importante, certi termini vanno imparati ed esposti nel modo in cui è stato spiegato e \textbf{non in maniera diversa} in quanto può comportare la perdita di punti all'esame.
\subsection{Esame}
L'esame è composto da un'interrogazione orale, gli argomenti d'esame sono gli stessi di quelli degli anni precedenti( questo vuol dire tutt'altro che non studiare). La parte dell'esame di diritto comporta $\frac{3}{9}$ dei crediti del corso, la verbalizzazione del voto verrà gestita da un singolo professore, (Prof. Pierotti al momento), gli altri professori non hanno potere a riguardo. I possessori di due voti positivi (di Diritto e di Economia) dovranno iscriversi alla \textbf{Verbalizzazione del voto}, pena l'\textbf{annullamento} del voto.  Ambe parti del voto sono valide fino al \textbf{termine dell'anno accademico}.
\subsection{Ricevimenti}
I ricevimenti possono avvenire sia brevemente a fine lezione in Ateneo, sia scrivendo all' \\ email \href{mailto:info@pclex.it}{info@pclex.it} richiedendo un appuntamento per discutere gli argomenti del corso.
\section{Principi Generali e Definizioni}

Normative sulla privacy (protezione dati): \begin{itemize}
    \item Regolamento (UE) 2016/679 in vigore dal 25 maggio 2018
    \item Direttiva 2002/58: Codice Europeo Comunicazioni Elettroniche
    \item D.Lgs. 196/2003: Abrogazione Selettiva.
    \item Provvedimenti Garante Privacy: \textbf{Non} decadono fino ad un eventuale modifica, sostituzione o abrogazione.
    \item Accordi internazionali su trasferimento dati: \textbf{Non} decadono fino ad un eventuale modifica, sostituzione o abrogazione.
\end{itemize}
Criteri direttivi: [missing] \newline
La Corte di Giustizia UE è la sede in cui un soggetto può far valere i propri diritti su direttive che sono state mal formalizzate disallineandosi a quanto definito dall' Unione Europea sulle leggi regolamentari riferite a \textbf{tutti} gli stati dell' UE.

Diverse leggi vengono prima idealizzate e solo ad una seconda data vengono implementate per permettere la comprensione in anticipo della legge.
L'abrogazione selettiva: [missing]
\newline GDPR: General data Protection and Regulation [missing] \newline

Alla creazione di nuove si studia quanto i suoi effetti possano abrogare (essere incompatibili) con precedenti leggi già in atto.
Bisogna prestare molta attenzione alla \textbf{successione temporale delle leggi} in quanto, come gli accordi internazionali degli stati, ci possono essere degli accordi specifici che non vengono abrogati in seguito ad una nuova legge.

Problema: I trasferimenti dei dati verso i server americani \textbf{non} erano compatibili alle normative europee. L'importante non è il confine fisico del server, anche se esiste una sede in uno stato UE, era \textbf{controllata da una casa madre} (\textcolor{red}{promanazione}), situata in America. I controlli americani iniziati dopo l'11 Settembre 2002 sono di tipo invasivo e quindi in contrasto con quanto voluto dall'UE, basata sui cittadini. A data  \today tutte le aziende americane sono obbligate a fornire \textbf{tutti i dati} degli utenti alle \textbf{intelligence} americane, comportamento \textbf{inaccettabile} per quanto detto dalle normative europee. Questo argomento si chiama: \textbf{\large Data Protection}.
\section{Accountability del Titolare}
\subsection{ Cos'è il principio di accountability?} è il perno della \textbf{metodologia} di tutta la gestione dei dati: Il titolare del trattamento dei dati è responsabile ai principi privacy e deve essere in grado di DIMOSTRARLA (art. 5, co.2)]
\subsection{Protezione sin dalla progettazione}
Esiste un articolo a doc. che disciplina la protezione dei dati sin dalla protegettazione: \begin{itemize}
    \item By design, le normative devono essere già applicate al momento della progettazione di un prodotto di un software.
    \item By default, (applicata anche nella realizzazione) prevede che vengano attuate delle misure per minimizzare il trattamento dei dati personali, quindi devono essere processati i dati personali necessari alla sua funzionalità.
\end{itemize}
\subsubsection{Obblighi Documentali}
\begin{itemize}
    \item Registro Trattamenti
    \item DPIA
    \item Lettere Incarico
    \item missing
\end{itemize}

\section{Applicabilità GDPR delle persone fisiche}
Il GDPR viene applicata solo sulle persone \textbf{fisiche}: ovvero gli esseri umani \textbf{vivi}. Le persone giuridiche sono gli enti che \textbf{non sono umani}, ma hanno una loro autonomia (comuni, Università...). I dati personali di persone giuridiche si applica la "Direttiva 2005/58".

\section{Definizione dati personali e termini relativi}


Cos'è un \textbf{dato personale}? Un dato personale è qualsiasi informazione riguardante una \textbf{persona fisica} identificata o identificabile (interessato); si considera identificabile la persona fisica che può essere identificata, direttamente o indirettamente, con particolare riferimento a un identificativo come il nome, un numero di identificazione.... 
\newline
\definition{\textbf{Trattamento}: (trattamento) qualsiasi operazione o insieme di operazioni, compiute con o senza l'ausilio di processi automatizzati e applicate a dati personali o insiemi di dati personali, come la raccolta, registrazione, organizzazione, strutturazione, conservazione, l'adattamento, modifica, la comunicazione mediante trasmissione, l'estrazione, diffusione, il raffronto, la limitazione, la cancellazione o la distruzione.}
\newline

Art. 9, comma 1 | \textbf{Dati ex Sensibili}: sono dati che rivelano l'origine razziale o etniche, le opinioni politiche, le convinzioni religiosi o filosofiche, l'appartenenza sindacale, dati biometrici, dati relativi alla salute, alla vita sessuale o all'orientamento sessuale della persona. (Ovvero \textbf{ALCUNI} dei dati che possono portare alla discriminazione).
\newline

Art. 10 | \textbf{Dati ex Giudiziari}: dati personali relativi alle condanne penali e ai reati o a connesse misure di sicurezza. (Penali)
\newline

Tutte le definizioni fondamentali possono essere trovate in un glossario apposito, \textbf{indicando l'intenzione di una ente di vedere formalmente tale parola in quel modo}.

\section{Soggetti che effettuano il Trattamento}
\definition{\textbf{Titolare del Trattamento}: La persona fisica o giuridica, l'autorità pubblica, il servizio o altro organismo che, singolarmente o insieme ad altri, \textit{determina le finalità e i mezzi del trattamento dati}; quando le finalità e i mezzi di tale trattamento sono determinati dal diritto dell'Unione Europea o degli stati membri, il titolare del trattamento o i criteri specifici applicabili alla sua designazione possono essere stabiliti dal diritto dell'Unione o degli Stati membri. }
\definition{\textbf{Responsabile del Trattamento}: La persona fisica o giuridica. l'autorità pubblica, il servizio o altro organismo che \textit{tratta i dati personali per conto del titolare del trattamento}.} Colui che tratta i dati personali senza essere definito responsabile dei dati, causa un \textbf{trattamento illecito dei dati}. Il contratto relativo si chiama: \textit{Contratto di Nomina del responsabile del Trattamento dei dati}. Documento trovabile su: \href{https://www.garanteprivacy.it/web/guest/home/docweb/-/docweb-display/docweb/7322273}{\textcolor{blue}{Garante Privacy}} (Attenzione prevede download istantaneo)

Articolo 28, comma 3, del Regolamento (UE) 2016/679: I trattamenti da parte di un responsabile del trattamento sono \textit{disciplinati da un \textbf{contratto} o da atto giuridico} a norma del diritto dell'Unione o degli Stati membri.
\definition{\textbf{Professionista}: Colui che ha le competenze per compiere la professione e le può dimostrare attraverso titolo di studio} (Il perito è considerato \textbf{Non Sufficiente}, viene esplicitato come sufficiente la \textbf{Laurea} o titolo equiparato).
\definition{\textbf{Incaricato}: Il responsabile del trattamento dei dati, o chiunque \textit{agisca sotto la sua autorità o sotto quella del titolare del trattamento}, che abbia accesso a dati personali \textit{non può trattare tali dati} se non è \textbf{istruito} in tal senso dal titolare del trattamento, salvo che lo richieda il diritto dell'Unione o degli Stati membri. }
\definition{\textbf{Interessato}: La persona fisica identificata o identificabile cui si riferiscono i dati personali. All'interessato si ricollegano [missing]}
\definition{\textbf{Responsabile Della Protezione dei Dati (DPO)}: Obbligatorio nei seguenti 3 casi: \begin{itemize}
    \item Autorità Pubblica o organismo [missing]
\end{itemize}}
\subsection{Diritti dell'Interessatto}
[missing]
\begin{itemize}
    \item Conoscitivi;
    \item Di controllo sul trattamento;
    \item Di intervento sui dati;
    \item Di non essere sottoposto a decisione basata \textbf{unicamente} sul trattamento automatizzato, compresa la profilazione, che produca effetti giuridici che lo riguardano o incida in modo analogo significativamente sulla sua persona (\href{https://gdpr-info.eu/art-22-gdpr/}{art 22}).
\end{itemize}
\definition{\textbf{ [missing]Autorità di Controllo in ambito GDPR privacy}: L'Autorità Garante ha funzioni di Controllo Normativo sulle materie di competenza nazionale. Organo collegiale composto da \textbf{quattro} membri eletti dal Parlamento della durata di 7 anni. Si articola in \begin{itemize}
    \item Segreteria Generale;
    \item Dipartimenti;
    \item Servizi;
\end{itemize}} (\href{https://www.brocardi.it/codice-della-privacy/parte-iii/titolo-ii/capo-i/art153.html}{art 153, comma 1})

\section{Informativa dell'Interessato}
Liceità del Trattamento.
\definition{\textbf{Informativa all'Interessato}: Unico documento sempre necessario da sottoporre al soggetto interessato contenente tutte le informazioni necessarie, deve essere scritto in forma concisa, trasparente, intelligibile e facilmente accessibile, con un linguaggio semplice e chiaro, \textit{specialmente destinate ai minori (di 14 anni)}. Le informazioni sono fornite per \textbf{iscritto} o con altri mezzi (elettronici); se richiesto dall'interessato, le informazioni possono essere \textit{fornite oralmente, purché sia comprovata con altri mezzi l'identità dell'interessato.} } In questa informativa deve essere scritto: \begin{itemize}
    \item I dati di contatto del responsabile della protezione dei dati;
    \item Le finalità del trattamento cui sono destinati i dati personali nonché la \textbf{base giuridica del trattamento};
    \item Gli eventuali \textbf{destinatari o le eventuali categorie di destinatari} dei dati personali;
    \item Ove applicabile, \textbf{l'intenzione del titolare del trattamento di trasferire dati personali a un paese terzo o a un'organizzazione internazionale};
    \item Il \textbf{periodo di conservazione} ed i criteri per determinare tale periodo.
    \item L'esistenza del diritto dell'interessato di chiedere al titolare del trattamento \textit{l'accesso ai dati personali} e la rettifica o la cancellazione degli stessi o la limitazione del trattamento che lo riguardano o di \textbf{opporsi al loro trattamento}, oltre al diritto alla portabilità dei dati.
    \item Qualora il trattamento sia basato sull'art 9, paragrafo 2, lettera a [missing]
\end{itemize}
Nota: Durante l'Agosto del 2022 Google analytics fece un utilizzo illecito dei dati personali non specificando i minimi legali verso gli interessati. 

Giurisdizione esclusiva mondiale: (america) [missing]

Il trattamento è \textbf{lecito} solo se è nella misura in cui ricorre almeno una delle condizioni seguenti:
\begin{itemize}
    \item L'interessato ha espresso il consenso al trattamento dei dati personali per una o più finalità (uno solo dei punti, non sempre è necessario il consenso, diverso da prenderne la visione).
    \item Il trattamento è necessario all'esecuzione di un contratto di cui l'interessato è parte.
    \item Il trattamento è necessario per adempiere ad un obbligo legale al quale è soggetto il titolare del trattamento;
    \item Il trattamento è necessario per la salvaguardia degli interessi vitali dell'interessato o di un'altra persona fisica.
    \item Il trattamento è necessario per l'esecuzione di un compito di interesse pubblico o connesso all'esercizio di pubblici poteri.
    \item Il trattamento è necessario per il perseguimento del \textbf{legittimo interesse} del titolare del trattamento o di terzi (proporre informazioni in base ad un mostrato interesse, tipo e-mail spam dopo certe ricerche).
\end{itemize}
\definition{\textbf{Consenso dell'interessato}: Il consenso mediante un \textit{atto positivo inequivocabile} con il quale l'interessato manifesta l'intenzione libera, specifica, informata e inequivocabile di accettare il trattamento dei dati personali che lo riguardano. 1) Il titolare deve essere in grado di dimostrare che l'interessato ha prestato il proprio consenso [missing] 3) L'interessato ha il diritto di revocare il proprio consenso in qualsiasi momento}

\section{Data Breach}
Secondo il WP29 il data breach consiste in una violazione di sicurezza che comporta accidentalmente o in modo illecito la distruzione, la perdita, la modifica, la divulgazione non autorizzata o l'accesso ai dati personali trasmessi, conservati o comunque trattati. Guida sulla notifica di Data Breach: ex art 29.

Secondo il WP29 la violazione dei dati personali si suddivide in tre categorie:\begin{itemize}
    \item Confidentiality Breach: Divulgazione o accesso non autorizzato ai dati;
    \item Availability Breach: Alterazione non autorizzata o accidentale di dati personali;
    \item Integrity Breach: Modifica non autorizzata o accidentale di dati personali.
\end{itemize}
\subsection{Notifica di violazione di dati}
Il titolare del trattamento dei notifica la violazione \textbf{all'autorità di controllo competenze} senza ingiustificato ritardo e ove possibile, \textbf{entro 72 ore dal momento in cui ne è venuto a conoscenza}, a meno che sia improbabile che la violazione dei dati personali presenti un \textbf{rischio per i diritti e le libertà delle persone fisiche} (Non serve notificare in caso i dati fossero cifrati o non è stato effettuata nessun accesso effettivo).

\textbf{La comunicazione all'interessato non è richiesta se:}
\begin{enumerate}
    \item Ha messo in atto le misure tecniche e organizzative adeguate di protezione e tali misure erano state applicate ai dati personali oggetto della violazione, in particolare quelle a rendere i dati illeggibili, a chiunque non sia autorizzato ad accedervi.
    \item Ha successivamente adottato misure atte a scongiurare il sopraggiungere di un rischio elevato per i diritti e le libertà degli interessato;
    \item Detta comunicazione richiederebbe \textbf{sforzi sproporzionati} (In caso di un attacco ad un soggetto che ha numerosi interessati può essere esonerato di comunicarlo ad uno ad uno e quindi usare delle metodologie che richiedono meno sforzi, ma che porti lo stesso all'interessato la comunicazione).
\end{enumerate}
\section{Il diritto Penale}
Il diritto penale è l'insieme di norme giuridiche con le quali lo stato proibisce, mediante la minaccia di una pena, [missing]
\newline La pena, invece, si può definire come una "sofferenza" che viene imposta dallo stato alla personale che ha violato un dovere giuridico. Il principale complesso di norme giuridiche è costituito dal \textbf{codice penale}, il cosiddetto \textbf{Codice Rocco} pubblicato nel 1930 è applicato dal 1931.
\subsection{Principio di Legalità}
\textit{"Nessuno può essere punito se non in forza di una legge che sia entrata in vigore prima del fatto commesso. Nessuno può essere sottoposto a misure di sicurezza se non nei casi previsti dalla legge"} \hyperlink{https://www.senato.it/istituzione/la-costituzione/parte-i/titolo-i/articolo-25}{(\textcolor{blue}{Art. 25 Costituzione Italiana})}
\begin{itemize}
    \item Riserva di Legge: solo al potere legislativo appartiene il monopolio normativo in materia penale.
    \item Tassatività: riguarda la tecnica di \textbf{formulazione} delle norme penali (precisa, vietando al giudice di creare analogie)
    \item Irretroattività: Irretroattività della legge sfavorevole, Retroattività della legge favorevole.
\end{itemize}
\subsection{Il Reato}
Solo la persona umana può commettere un reato, si divide in \begin{itemize}
    \item \textbf{Capacità Penale}: riguarda tutti indistintamente;
    \item \textbf{Capacità alla Pena}: presuppone l'imputabilità;
\end{itemize}
\subsection{Precetto e Sanzione}
Il \textbf{Precetto} è il comando di tenere una certa condotta, ad esempio \textit{"Chiunque cagiona la morte di un uomo è punito con la reclusione"}, senza nessuna specifica. La \textbf{Sanzione} è la \textbf{conseguenza giuridica che deve seguire l'infrazione del precetto}. \newline In alcuni casi il legislatore affida la descrizione del precetto a fonti extra penali, ossia a norme che provengono da altri rami dell'ordinamento. Un ulteriore caso è dove il legislatore \textbf{rimanda} l'integrazione del precetto ad atti normativi secondari o, addirittura, ad atti non normativi; un esempio: il decreto del Ministero della Sanità sull'aggiornamento delle tabelle degli \textbf{stupefacenti}.
\subsection{Bene Giuridico}
Ogni norma penale tutela un determinato bene o interesse, ovvero è il \textbf{bene giuridico o l'interesse  giuridico tutelato dalla norma}. I beni protetti sono individuati dai \textbf{principi sanciti dalla costituzione }. \textbf{Nota:} non da confondere con \textbf{l'oggetto materiale dell'azione}.
\subsection{Condotta}




\end{document}
